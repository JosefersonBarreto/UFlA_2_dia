% Options for packages loaded elsewhere
\PassOptionsToPackage{unicode}{hyperref}
\PassOptionsToPackage{hyphens}{url}
\PassOptionsToPackage{dvipsnames,svgnames,x11names}{xcolor}
%
\documentclass[
  letterpaper,
  DIV=11,
  numbers=noendperiod]{scrartcl}

\usepackage{amsmath,amssymb}
\usepackage{iftex}
\ifPDFTeX
  \usepackage[T1]{fontenc}
  \usepackage[utf8]{inputenc}
  \usepackage{textcomp} % provide euro and other symbols
\else % if luatex or xetex
  \usepackage{unicode-math}
  \defaultfontfeatures{Scale=MatchLowercase}
  \defaultfontfeatures[\rmfamily]{Ligatures=TeX,Scale=1}
\fi
\usepackage{lmodern}
\ifPDFTeX\else  
    % xetex/luatex font selection
\fi
% Use upquote if available, for straight quotes in verbatim environments
\IfFileExists{upquote.sty}{\usepackage{upquote}}{}
\IfFileExists{microtype.sty}{% use microtype if available
  \usepackage[]{microtype}
  \UseMicrotypeSet[protrusion]{basicmath} % disable protrusion for tt fonts
}{}
\makeatletter
\@ifundefined{KOMAClassName}{% if non-KOMA class
  \IfFileExists{parskip.sty}{%
    \usepackage{parskip}
  }{% else
    \setlength{\parindent}{0pt}
    \setlength{\parskip}{6pt plus 2pt minus 1pt}}
}{% if KOMA class
  \KOMAoptions{parskip=half}}
\makeatother
\usepackage{xcolor}
\setlength{\emergencystretch}{3em} % prevent overfull lines
\setcounter{secnumdepth}{-\maxdimen} % remove section numbering
% Make \paragraph and \subparagraph free-standing
\ifx\paragraph\undefined\else
  \let\oldparagraph\paragraph
  \renewcommand{\paragraph}[1]{\oldparagraph{#1}\mbox{}}
\fi
\ifx\subparagraph\undefined\else
  \let\oldsubparagraph\subparagraph
  \renewcommand{\subparagraph}[1]{\oldsubparagraph{#1}\mbox{}}
\fi

\usepackage{color}
\usepackage{fancyvrb}
\newcommand{\VerbBar}{|}
\newcommand{\VERB}{\Verb[commandchars=\\\{\}]}
\DefineVerbatimEnvironment{Highlighting}{Verbatim}{commandchars=\\\{\}}
% Add ',fontsize=\small' for more characters per line
\usepackage{framed}
\definecolor{shadecolor}{RGB}{241,243,245}
\newenvironment{Shaded}{\begin{snugshade}}{\end{snugshade}}
\newcommand{\AlertTok}[1]{\textcolor[rgb]{0.68,0.00,0.00}{#1}}
\newcommand{\AnnotationTok}[1]{\textcolor[rgb]{0.37,0.37,0.37}{#1}}
\newcommand{\AttributeTok}[1]{\textcolor[rgb]{0.40,0.45,0.13}{#1}}
\newcommand{\BaseNTok}[1]{\textcolor[rgb]{0.68,0.00,0.00}{#1}}
\newcommand{\BuiltInTok}[1]{\textcolor[rgb]{0.00,0.23,0.31}{#1}}
\newcommand{\CharTok}[1]{\textcolor[rgb]{0.13,0.47,0.30}{#1}}
\newcommand{\CommentTok}[1]{\textcolor[rgb]{0.37,0.37,0.37}{#1}}
\newcommand{\CommentVarTok}[1]{\textcolor[rgb]{0.37,0.37,0.37}{\textit{#1}}}
\newcommand{\ConstantTok}[1]{\textcolor[rgb]{0.56,0.35,0.01}{#1}}
\newcommand{\ControlFlowTok}[1]{\textcolor[rgb]{0.00,0.23,0.31}{#1}}
\newcommand{\DataTypeTok}[1]{\textcolor[rgb]{0.68,0.00,0.00}{#1}}
\newcommand{\DecValTok}[1]{\textcolor[rgb]{0.68,0.00,0.00}{#1}}
\newcommand{\DocumentationTok}[1]{\textcolor[rgb]{0.37,0.37,0.37}{\textit{#1}}}
\newcommand{\ErrorTok}[1]{\textcolor[rgb]{0.68,0.00,0.00}{#1}}
\newcommand{\ExtensionTok}[1]{\textcolor[rgb]{0.00,0.23,0.31}{#1}}
\newcommand{\FloatTok}[1]{\textcolor[rgb]{0.68,0.00,0.00}{#1}}
\newcommand{\FunctionTok}[1]{\textcolor[rgb]{0.28,0.35,0.67}{#1}}
\newcommand{\ImportTok}[1]{\textcolor[rgb]{0.00,0.46,0.62}{#1}}
\newcommand{\InformationTok}[1]{\textcolor[rgb]{0.37,0.37,0.37}{#1}}
\newcommand{\KeywordTok}[1]{\textcolor[rgb]{0.00,0.23,0.31}{#1}}
\newcommand{\NormalTok}[1]{\textcolor[rgb]{0.00,0.23,0.31}{#1}}
\newcommand{\OperatorTok}[1]{\textcolor[rgb]{0.37,0.37,0.37}{#1}}
\newcommand{\OtherTok}[1]{\textcolor[rgb]{0.00,0.23,0.31}{#1}}
\newcommand{\PreprocessorTok}[1]{\textcolor[rgb]{0.68,0.00,0.00}{#1}}
\newcommand{\RegionMarkerTok}[1]{\textcolor[rgb]{0.00,0.23,0.31}{#1}}
\newcommand{\SpecialCharTok}[1]{\textcolor[rgb]{0.37,0.37,0.37}{#1}}
\newcommand{\SpecialStringTok}[1]{\textcolor[rgb]{0.13,0.47,0.30}{#1}}
\newcommand{\StringTok}[1]{\textcolor[rgb]{0.13,0.47,0.30}{#1}}
\newcommand{\VariableTok}[1]{\textcolor[rgb]{0.07,0.07,0.07}{#1}}
\newcommand{\VerbatimStringTok}[1]{\textcolor[rgb]{0.13,0.47,0.30}{#1}}
\newcommand{\WarningTok}[1]{\textcolor[rgb]{0.37,0.37,0.37}{\textit{#1}}}

\providecommand{\tightlist}{%
  \setlength{\itemsep}{0pt}\setlength{\parskip}{0pt}}\usepackage{longtable,booktabs,array}
\usepackage{calc} % for calculating minipage widths
% Correct order of tables after \paragraph or \subparagraph
\usepackage{etoolbox}
\makeatletter
\patchcmd\longtable{\par}{\if@noskipsec\mbox{}\fi\par}{}{}
\makeatother
% Allow footnotes in longtable head/foot
\IfFileExists{footnotehyper.sty}{\usepackage{footnotehyper}}{\usepackage{footnote}}
\makesavenoteenv{longtable}
\usepackage{graphicx}
\makeatletter
\def\maxwidth{\ifdim\Gin@nat@width>\linewidth\linewidth\else\Gin@nat@width\fi}
\def\maxheight{\ifdim\Gin@nat@height>\textheight\textheight\else\Gin@nat@height\fi}
\makeatother
% Scale images if necessary, so that they will not overflow the page
% margins by default, and it is still possible to overwrite the defaults
% using explicit options in \includegraphics[width, height, ...]{}
\setkeys{Gin}{width=\maxwidth,height=\maxheight,keepaspectratio}
% Set default figure placement to htbp
\makeatletter
\def\fps@figure{htbp}
\makeatother
% definitions for citeproc citations
\NewDocumentCommand\citeproctext{}{}
\NewDocumentCommand\citeproc{mm}{%
  \begingroup\def\citeproctext{#2}\cite{#1}\endgroup}
\makeatletter
 % allow citations to break across lines
 \let\@cite@ofmt\@firstofone
 % avoid brackets around text for \cite:
 \def\@biblabel#1{}
 \def\@cite#1#2{{#1\if@tempswa , #2\fi}}
\makeatother
\newlength{\cslhangindent}
\setlength{\cslhangindent}{1.5em}
\newlength{\csllabelwidth}
\setlength{\csllabelwidth}{3em}
\newenvironment{CSLReferences}[2] % #1 hanging-indent, #2 entry-spacing
 {\begin{list}{}{%
  \setlength{\itemindent}{0pt}
  \setlength{\leftmargin}{0pt}
  \setlength{\parsep}{0pt}
  % turn on hanging indent if param 1 is 1
  \ifodd #1
   \setlength{\leftmargin}{\cslhangindent}
   \setlength{\itemindent}{-1\cslhangindent}
  \fi
  % set entry spacing
  \setlength{\itemsep}{#2\baselineskip}}}
 {\end{list}}
\usepackage{calc}
\newcommand{\CSLBlock}[1]{\hfill\break\parbox[t]{\linewidth}{\strut\ignorespaces#1\strut}}
\newcommand{\CSLLeftMargin}[1]{\parbox[t]{\csllabelwidth}{\strut#1\strut}}
\newcommand{\CSLRightInline}[1]{\parbox[t]{\linewidth - \csllabelwidth}{\strut#1\strut}}
\newcommand{\CSLIndent}[1]{\hspace{\cslhangindent}#1}

\KOMAoption{captions}{tableheading}
\makeatletter
\@ifpackageloaded{caption}{}{\usepackage{caption}}
\AtBeginDocument{%
\ifdefined\contentsname
  \renewcommand*\contentsname{Índice}
\else
  \newcommand\contentsname{Índice}
\fi
\ifdefined\listfigurename
  \renewcommand*\listfigurename{Lista de Figuras}
\else
  \newcommand\listfigurename{Lista de Figuras}
\fi
\ifdefined\listtablename
  \renewcommand*\listtablename{Lista de Tabelas}
\else
  \newcommand\listtablename{Lista de Tabelas}
\fi
\ifdefined\figurename
  \renewcommand*\figurename{Figura}
\else
  \newcommand\figurename{Figura}
\fi
\ifdefined\tablename
  \renewcommand*\tablename{Tabela}
\else
  \newcommand\tablename{Tabela}
\fi
}
\@ifpackageloaded{float}{}{\usepackage{float}}
\floatstyle{ruled}
\@ifundefined{c@chapter}{\newfloat{codelisting}{h}{lop}}{\newfloat{codelisting}{h}{lop}[chapter]}
\floatname{codelisting}{Listagem}
\newcommand*\listoflistings{\listof{codelisting}{Lista de Listagens}}
\makeatother
\makeatletter
\makeatother
\makeatletter
\@ifpackageloaded{caption}{}{\usepackage{caption}}
\@ifpackageloaded{subcaption}{}{\usepackage{subcaption}}
\makeatother
\ifLuaTeX
\usepackage[bidi=basic]{babel}
\else
\usepackage[bidi=default]{babel}
\fi
\babelprovide[main,import]{portuguese}
% get rid of language-specific shorthands (see #6817):
\let\LanguageShortHands\languageshorthands
\def\languageshorthands#1{}
\ifLuaTeX
  \usepackage{selnolig}  % disable illegal ligatures
\fi
\usepackage{bookmark}

\IfFileExists{xurl.sty}{\usepackage{xurl}}{} % add URL line breaks if available
\urlstyle{same} % disable monospaced font for URLs
\hypersetup{
  pdftitle={   Análise Exploratória da produção de Leite de Três Vacas em uma fazenda},
  pdfauthor={joseferson barreto, bacharel em estatística pela úniversidade Estadual da Paraíba; Dr.~Thiago Almeida, Universidade Estadual da Paraíba},
  pdflang={pt},
  colorlinks=true,
  linkcolor={blue},
  filecolor={Maroon},
  citecolor={Blue},
  urlcolor={Blue},
  pdfcreator={LaTeX via pandoc}}

\title{\\
\strut \\
\strut \\
Análise Exploratória da produção de Leite de Três Vacas em uma fazenda}
\author{joseferson barreto, bacharel em estatística pela úniversidade
Estadual da Paraíba \and Dr.~Thiago Almeida, Universidade Estadual da
Paraíba}
\date{}

\begin{document}
\maketitle

\renewcommand*\contentsname{Sumário}
{
\hypersetup{linkcolor=}
\setcounter{tocdepth}{3}
\tableofcontents
}
\section{Carregando Algumas
Bibliotecas}\label{carregando-algumas-bibliotecas}

Primeiramente vamos carregar algumas bibliotecas que usaremos em nossas
análises .

\begin{itemize}
\item
  A biblioteca \textbf{readxl} Wickham e Bryan (2023) para ler o
  conjunto de dados do excel.
\item
  A biblioteca \textbf{dplyr} Wickham et al. (2023) para algumas
  manipulações no conjunto de dados.
\item
  A biblioteca \textbf{knitr} Xie (2023) para manipulações com latex e
  afins.
\item
  A biblioteca \textbf{ggplot2} Wickham (2016) para geração de gráficos.
\end{itemize}

\begin{Shaded}
\begin{Highlighting}[]
\CommentTok{\#install.packages("shinythemes")}
\FunctionTok{library}\NormalTok{(readxl)}
\FunctionTok{library}\NormalTok{(dplyr)}
\CommentTok{\#getwd()}
\FunctionTok{library}\NormalTok{(knitr)}
\end{Highlighting}
\end{Shaded}

\subsection{Carregando o Banco de
Dados}\label{carregando-o-banco-de-dados}

Carregando o banco de dados referente a produção de leite de 3 vacas em
uma fazenda

\begin{Shaded}
\begin{Highlighting}[]
\CommentTok{\#teste\textless{}{-}read.table("C:/Users/joseferson/Documents/joseferson barreto/projeto{-}final{-}planejamento2/psubDBC.txt")}
\NormalTok{ dados }\OtherTok{\textless{}{-}} \FunctionTok{read\_excel}\NormalTok{(}\StringTok{"dados.xlsx"}\NormalTok{) }
\NormalTok{dados}\OtherTok{\textless{}{-}}\NormalTok{ dados[}\SpecialCharTok{{-}}\DecValTok{1}\NormalTok{]}
 \FunctionTok{library}\NormalTok{(knitr)}
\FunctionTok{kable}\NormalTok{(}\FunctionTok{head}\NormalTok{(dados, }\DecValTok{10}\NormalTok{))}
\end{Highlighting}
\end{Shaded}

\begin{longtable}[]{@{}lllrrr@{}}
\toprule\noalign{}
racao\_Comun & Turno & mes & pasta\_bovin & soja & rep \\
\midrule\noalign{}
\endhead
\bottomrule\noalign{}
\endlastfoot
26 & dia & primeiro mês & 41 & 41 & 1 \\
26 & tarde & primeiro mês & 41 & 41 & 1 \\
33 & dia & primeiro mês & 37 & 39 & 1 \\
25 & tarde & primeiro mês & 40 & 37 & 1 \\
29 & dia & primeiro mês & 37 & 39 & 1 \\
34 & tarde & primeiro mês & 36 & 40 & 1 \\
28 & dia & primeiro mês & 36 & 38 & 1 \\
27 & tarde & primeiro mês & 40 & 41 & 1 \\
29 & dia & primeiro mês & 37 & 37 & 1 \\
32 & tarde & primeiro mês & 39 & 43 & 1 \\
\end{longtable}

\begin{Shaded}
\begin{Highlighting}[]
\CommentTok{\#==================================================================================}
\CommentTok{\#                      transformando os dados                                    \#}
\CommentTok{\#==================================================================================}


\CommentTok{\# Criar a nova variável "dia\_mes"}
\NormalTok{dados }\OtherTok{\textless{}{-}}\NormalTok{ dados }\SpecialCharTok{\%\textgreater{}\%}
  \FunctionTok{group\_by}\NormalTok{(mes) }\SpecialCharTok{\%\textgreater{}\%}
  \FunctionTok{mutate}\NormalTok{(}
    \AttributeTok{dia\_mes =} \FunctionTok{rep}\NormalTok{(}\FunctionTok{paste0}\NormalTok{(}\StringTok{"dia "}\NormalTok{, }\FunctionTok{rep}\NormalTok{(}\DecValTok{1}\SpecialCharTok{:}\DecValTok{30}\NormalTok{, }\AttributeTok{each =} \DecValTok{2}\NormalTok{)), }\AttributeTok{length.out =} \FunctionTok{n}\NormalTok{())}
\NormalTok{  )}

\CommentTok{\# Visualizar os dados resultantes}
\CommentTok{\#print(dados)}



\CommentTok{\# Criar a nova variável "dia\_mes\_com\_mes"}
\NormalTok{dados }\OtherTok{\textless{}{-}}\NormalTok{ dados }\SpecialCharTok{\%\textgreater{}\%}
  \FunctionTok{mutate}\NormalTok{(}\AttributeTok{dia\_mes\_com\_mes =} \FunctionTok{paste}\NormalTok{(dia\_mes, mes, }\AttributeTok{sep =} \StringTok{" "}\NormalTok{))}

\CommentTok{\# Criando vetores vazios para armazenar os dados e a identificação da variável}
\NormalTok{prod }\OtherTok{\textless{}{-}} \FunctionTok{c}\NormalTok{()}
\NormalTok{variavel }\OtherTok{\textless{}{-}} \FunctionTok{c}\NormalTok{()}

\CommentTok{\# Número total de observações e de variáveis}
\NormalTok{num\_observacoes }\OtherTok{\textless{}{-}} \DecValTok{360}
\NormalTok{variaveis }\OtherTok{\textless{}{-}} \FunctionTok{c}\NormalTok{(}\StringTok{"soja"}\NormalTok{, }\StringTok{"pasta\_bovin"}\NormalTok{, }\StringTok{"racao\_Comun"}\NormalTok{)}
\NormalTok{num\_variaveis }\OtherTok{\textless{}{-}} \FunctionTok{length}\NormalTok{(variaveis)}

\CommentTok{\# Loop para preencher as variáveis prod e variavel}
\ControlFlowTok{for}\NormalTok{ (i }\ControlFlowTok{in} \DecValTok{1}\SpecialCharTok{:}\NormalTok{(num\_observacoes}\SpecialCharTok{/}\DecValTok{2}\NormalTok{)) \{}
  \CommentTok{\# Iteração sobre as variáveis}
  \ControlFlowTok{for}\NormalTok{ (j }\ControlFlowTok{in} \DecValTok{1}\SpecialCharTok{:}\NormalTok{num\_variaveis) \{}
    \CommentTok{\# Adicionando duas observações da variável atual}
\NormalTok{    prod }\OtherTok{\textless{}{-}} \FunctionTok{c}\NormalTok{(prod, dados[[variaveis[j]]][((i }\SpecialCharTok{{-}} \DecValTok{1}\NormalTok{) }\SpecialCharTok{*} \DecValTok{2} \SpecialCharTok{+} \DecValTok{1}\NormalTok{)}\SpecialCharTok{:}\NormalTok{(i }\SpecialCharTok{*} \DecValTok{2}\NormalTok{)])}
    
    \CommentTok{\# Adicionando a identificação da variável}
\NormalTok{    variavel }\OtherTok{\textless{}{-}} \FunctionTok{c}\NormalTok{(variavel, }\FunctionTok{rep}\NormalTok{(variaveis[j], }\DecValTok{2}\NormalTok{))}
\NormalTok{  \}}
\NormalTok{\}}

\CommentTok{\# \# Verificando os resultados}
\CommentTok{\# head(prod)}
\CommentTok{\# head(variavel)}






\NormalTok{banco}\OtherTok{\textless{}{-}}\FunctionTok{data.frame}\NormalTok{(}\AttributeTok{prod=}\FunctionTok{c}\NormalTok{(prod),}\AttributeTok{turno=}\FunctionTok{as.factor}\NormalTok{(}\FunctionTok{c}\NormalTok{(dados}\SpecialCharTok{$}\NormalTok{Turno,}
\NormalTok{              dados}\SpecialCharTok{$}\NormalTok{Turno,dados}\SpecialCharTok{$}\NormalTok{Turno)),}\AttributeTok{blocos=}\FunctionTok{as.factor}\NormalTok{(}\FunctionTok{c}\NormalTok{(dados}\SpecialCharTok{$}\NormalTok{rep,}
\NormalTok{                  dados}\SpecialCharTok{$}\NormalTok{rep,dados}\SpecialCharTok{$}\NormalTok{rep))) }\CommentTok{\#, dias = c(dados$dia\_mes\_com\_mes,}
                 \CommentTok{\# dados$dia\_mes\_com\_mes,dados$dia\_mes\_com\_mes))}



\NormalTok{banco}\SpecialCharTok{$}\NormalTok{tipos\_rações}\OtherTok{\textless{}{-}}\FunctionTok{as.factor}\NormalTok{(}\FunctionTok{c}\NormalTok{(variavel))}



\NormalTok{banco}\SpecialCharTok{$}\NormalTok{prod}\OtherTok{\textless{}{-}}\FunctionTok{as.numeric}\NormalTok{(banco}\SpecialCharTok{$}\NormalTok{prod)}
\end{Highlighting}
\end{Shaded}

\textbf{Observação}

para mais temas e costomisações confira a documentação da biblioteca
\textbf{shinythemes} Chang (2021)

\subsection{Exibindo o Dataset Final}\label{exibindo-o-dataset-final}

\begin{Shaded}
\begin{Highlighting}[]
\NormalTok{p}\OtherTok{\textless{}{-}}\NormalTok{banco }\SpecialCharTok{|\textgreater{}} \FunctionTok{select}\NormalTok{(tipos\_rações,turno,blocos,prod) }
\NormalTok{banco }\OtherTok{\textless{}{-}}\NormalTok{p}
\NormalTok{banco1 }\OtherTok{\textless{}{-}}\NormalTok{ banco}

\FunctionTok{colnames}\NormalTok{(banco1)}\OtherTok{\textless{}{-}}\FunctionTok{c}\NormalTok{(}\StringTok{"tipo rações"}\NormalTok{,}\StringTok{"Turno"}\NormalTok{,}\StringTok{"Bloco"}\NormalTok{, }\StringTok{\textquotesingle{}prod\textquotesingle{}}\NormalTok{)}

\FunctionTok{kable}\NormalTok{(}\FunctionTok{head}\NormalTok{(banco1, }\DecValTok{10}\NormalTok{))}
\end{Highlighting}
\end{Shaded}

\begin{longtable}[]{@{}lllr@{}}
\toprule\noalign{}
tipo rações & Turno & Bloco & prod \\
\midrule\noalign{}
\endhead
\bottomrule\noalign{}
\endlastfoot
soja & dia & 1 & 41 \\
soja & tarde & 1 & 41 \\
pasta\_bovin & dia & 1 & 41 \\
pasta\_bovin & tarde & 1 & 41 \\
racao\_Comun & dia & 1 & 26 \\
racao\_Comun & tarde & 1 & 26 \\
soja & dia & 1 & 39 \\
soja & tarde & 1 & 37 \\
pasta\_bovin & dia & 1 & 37 \\
pasta\_bovin & tarde & 1 & 40 \\
\end{longtable}

\begin{longtable}[]{@{}lllr@{}}
\caption{fonte: Joseferson Barreto}\tabularnewline
\toprule\noalign{}
tipo rações & Turno & Bloco & prod \\
\midrule\noalign{}
\endfirsthead
\toprule\noalign{}
tipo rações & Turno & Bloco & prod \\
\midrule\noalign{}
\endhead
\bottomrule\noalign{}
\endlastfoot
soja & dia & 1 & 41 \\
soja & tarde & 1 & 41 \\
pasta\_bovin & dia & 1 & 41 \\
pasta\_bovin & tarde & 1 & 41 \\
racao\_Comun & dia & 1 & 26 \\
racao\_Comun & tarde & 1 & 26 \\
soja & dia & 1 & 39 \\
soja & tarde & 1 & 37 \\
pasta\_bovin & dia & 1 & 37 \\
pasta\_bovin & tarde & 1 & 40 \\
\end{longtable}

\subsection{Fazendo a análise
exploratória}\label{fazendo-a-anuxe1lise-exploratuxf3ria}

Vamos fazer a análise exploratória,primeiramente vamos observar osn
tipos de variáveis

\begin{Shaded}
\begin{Highlighting}[]
\CommentTok{\#| label: fig{-}pressure}
\CommentTok{\#| fig{-}cap: "Pressure"}
\CommentTok{\#| code{-}fold: true}

\CommentTok{\# converter  a variável prod para númerica }

\NormalTok{banco}\SpecialCharTok{$}\NormalTok{prod }\OtherTok{\textless{}{-}} \FunctionTok{as.integer}\NormalTok{(banco}\SpecialCharTok{$}\NormalTok{prod)}


\NormalTok{banco }\SpecialCharTok{|\textgreater{}} \FunctionTok{glimpse}\NormalTok{()}
\end{Highlighting}
\end{Shaded}

\begin{verbatim}
Rows: 1,080
Columns: 4
$ tipos_rações <fct> soja, soja, pasta_bovin, pasta_bovin, racao_Comun, racao_~
$ turno        <fct> dia, tarde, dia, tarde, dia, tarde, dia, tarde, dia, tard~
$ blocos       <fct> 1, 1, 1, 1, 1, 1, 1, 1, 1, 1, 1, 1, 1, 1, 1, 1, 1, 1, 1, ~
$ prod         <int> 41, 41, 41, 41, 26, 26, 39, 37, 37, 40, 33, 25, 39, 40, 3~
\end{verbatim}

\begin{Shaded}
\begin{Highlighting}[]
\CommentTok{\#| label: fig{-}pressure}
\CommentTok{\#| fig{-}cap: "Pressure"}
\CommentTok{\#| code{-}fold: true}



\CommentTok{\# library(ggplot2)}
\CommentTok{\# }
\CommentTok{\# ggplot(data = banco, aes(x = turno,y = tipos\_rações)) +}
\CommentTok{\#   geom\_bar() +}
\CommentTok{\#   labs(title = "Distribuição da Variável Turno",}
\CommentTok{\#        x = "Turno",}
\CommentTok{\#        y = "Contagem")}
\CommentTok{\# }
\CommentTok{\# }
\CommentTok{\# }
\CommentTok{\# ggplot(data = banco, aes(x = turno, fill = tipos\_rações)) +}
\CommentTok{\#   group\_by(tipos\_rações)+}
\CommentTok{\#   geom\_bar(position = "stack") +}
\CommentTok{\#   labs(title = "Distribuição de Turno por Tipos de Rações",}
\CommentTok{\#        x = "Turno",}
\CommentTok{\#        y = "Contagem",}
\CommentTok{\#        fill = "Tipos de Rações")}
\CommentTok{\# }
\CommentTok{\# }
\CommentTok{\# }
\CommentTok{\# }
\CommentTok{\# }
\CommentTok{\# }
\CommentTok{\# }
\CommentTok{\# }
\CommentTok{\# ggplot(data = banco, aes(x = turno, fill = tipos\_rações)) +}
\CommentTok{\#   geom\_bar(position = "stack") +}
\CommentTok{\#   labs(title = "Distribuição de Turno por Tipos de Rações",}
\CommentTok{\#        x = "Turno",}
\CommentTok{\#        y = "Contagem",}
\CommentTok{\#        fill = "Tipos de Rações")}

\CommentTok{\#install.packages("moments",dependencies = TRUE)}
\FunctionTok{library}\NormalTok{(moments)}

\NormalTok{descritiva }\OtherTok{\textless{}{-}}\NormalTok{ banco }\SpecialCharTok{\%\textgreater{}\%}
  \FunctionTok{summarise}\NormalTok{(}\StringTok{"Média produção"} \OtherTok{=} \FunctionTok{sprintf}\NormalTok{(}\StringTok{"\%.2f"}\NormalTok{, }\FunctionTok{round}\NormalTok{(}\FunctionTok{mean}\NormalTok{(prod, }\AttributeTok{na.rm =} \ConstantTok{TRUE}\NormalTok{), }\DecValTok{2}\NormalTok{)),}
            \StringTok{"Mediana produção"} \OtherTok{=} \FunctionTok{sprintf}\NormalTok{(}\StringTok{"\%.2f"}\NormalTok{, }\FunctionTok{round}\NormalTok{(}\FunctionTok{median}\NormalTok{(prod, }\AttributeTok{na.rm =} \ConstantTok{TRUE}\NormalTok{), }\DecValTok{2}\NormalTok{)),}
            \StringTok{"Assimétria"} \OtherTok{=} \FunctionTok{sprintf}\NormalTok{(}\StringTok{"\%.2f"}\NormalTok{, }\FunctionTok{round}\NormalTok{(}\FunctionTok{skewness}\NormalTok{(prod, }\AttributeTok{na.rm =} \ConstantTok{TRUE}\NormalTok{), }\DecValTok{2}\NormalTok{)),}
            \StringTok{"Curtose"} \OtherTok{=} \FunctionTok{sprintf}\NormalTok{(}\StringTok{"\%.2f"}\NormalTok{, }\FunctionTok{round}\NormalTok{(}\FunctionTok{kurtosis}\NormalTok{(prod, }\AttributeTok{na.rm =} \ConstantTok{TRUE}\NormalTok{), }\DecValTok{2}\NormalTok{)))}

\CommentTok{\#============================================================================}
\CommentTok{\#                   observação}
\CommentTok{\#============================================================================}


\CommentTok{\# A função sprintf("\%.2f") formata a saida dos resultados para que os zeros após a virgula sejam exibidos }


\NormalTok{descritiva }\SpecialCharTok{|\textgreater{}} \FunctionTok{kable}\NormalTok{(}\AttributeTok{align =} \StringTok{"c"}\NormalTok{)}
\end{Highlighting}
\end{Shaded}

\begin{longtable}[]{@{}cccc@{}}
\toprule\noalign{}
Média produção & Mediana produção & Assimétria & Curtose \\
\midrule\noalign{}
\endhead
\bottomrule\noalign{}
\endlastfoot
24.50 & 28.00 & -0.28 & 1.67 \\
\end{longtable}

\subsubsection{Interpretando os
Resultados}\label{interpretando-os-resultados}

\begin{itemize}
\item
  \emph{Média}

  \begin{itemize}
  \tightlist
  \item
    Logo , podemos afirmar que a média da produção de leite dessas 3
    vagas ao longo dos 6 meses foi de 24,50 litros
  \end{itemize}
\item
  \emph{Mediana}

  \begin{itemize}
  \tightlist
  \item
    A mediana da produção de leite dessas 3 vagas juntas foi de 28
    litros ao longo desses 6 meses
  \end{itemize}
\end{itemize}

\emph{Assimétria}

\begin{itemize}
\tightlist
\item
  Note que o coeficiente de assimétria foi de -0,28 , isso indica que
  temos uma ,\textbf{Assimetria negativa}: A distribuição é inclinada
  para a esquerda, indicando que há uma dispersão maior de valores
  menores em relação aos valores maiores.
\end{itemize}

Exemplos comuns de distribuições assimétricas à esquerda incluem
distribuições de tempo de espera em uma fila, distribuições de tempo de
vida de produtos.

\begin{itemize}
\item
  \emph{Curtose}

  \begin{itemize}
  \tightlist
  \item
    o Valor da curtose em nosso conjunto de dados foi de 1,67.
  \end{itemize}

  A Curtose é uma medida estatística que descreve a forma da
  distribuição de probabilidade de uma variável aleatória. Ela compara a
  distribuição de uma variável com a distribuição normal padrão. A
  curtose de uma distribuição normal padrão é definida como 3.

  Se \(K >3\) a distribuição é mais ``pontiaguda'' e ``pesada'' nas
  caudas do que uma distribuição normal, e é chamada de leptocúrtica.

  Se \(K < 3\) a distribuição é mais ``achatada'' do que uma
  distribuição normal, e é chamada de platicúrtica.

  Se \(K = 3\) a distribuição é chamada de mesocúrtica e tem a mesma
  curtose da distribuição normal.

  Logo, podemos afirmar que nossa siatribuição é mais achatada que a
  distribuição normal, oe seja , ela é platicúrtica.
\item
  \emph{Moda}
\end{itemize}

A moda é lor que ocorre com mais frequência em uma distribuição de
dados. Vamos utilizar o código abaixo para calcular a moda:

\begin{Shaded}
\begin{Highlighting}[]
\CommentTok{\#| label: fig{-}pressure}
\CommentTok{\#| fig{-}cap: "Pressure"}
\CommentTok{\#| code{-}fold: true}



\NormalTok{banco }\SpecialCharTok{|\textgreater{}}  \FunctionTok{select}\NormalTok{(prod)  }\SpecialCharTok{\%\textgreater{}\%} 

                 \FunctionTok{table}\NormalTok{()     }\SpecialCharTok{\%\textgreater{}\%}

                 \FunctionTok{which.max}\NormalTok{ () }\SpecialCharTok{\%\textgreater{}\%} 

                 \FunctionTok{names}\NormalTok{ ()     }\SpecialCharTok{\%\textgreater{}\%}  

                 \FunctionTok{as.numeric}\NormalTok{()}
\end{Highlighting}
\end{Shaded}

\begin{verbatim}
[1] 35
\end{verbatim}

Note que a moda foi maior que a média e a mediana ou seja
\(\text{Moda} > \text{Mediana} > \text{Média}\) isso indica que temos
uma Assimétria negátiva como vimos anteriomente .

Logo, com essas características apresentadas pode-se dizer que nossa
variável resposta não segue a distribução normal,vamos mostrar em um
histograma

\begin{Shaded}
\begin{Highlighting}[]
\CommentTok{\#| label: fig{-}pressure}
\CommentTok{\#| fig{-}cap: "Pressure"}
\CommentTok{\#| code{-}fold: true}


\CommentTok{\#hist(banco$prod)}
\FunctionTok{library}\NormalTok{(dplyr)}
\FunctionTok{library}\NormalTok{(ggplot2)}

\NormalTok{banco\_summary }\OtherTok{\textless{}{-}}\NormalTok{ banco }\SpecialCharTok{\%\textgreater{}\%}
  \FunctionTok{group\_by}\NormalTok{(turno, tipos\_rações) }\SpecialCharTok{\%\textgreater{}\%}
  \FunctionTok{summarize}\NormalTok{(}\AttributeTok{Contagem =} \FunctionTok{n}\NormalTok{())}

\CommentTok{\# ggplot(data = banco\_summary, aes(x = turno, y = Contagem, fill = tipos\_rações)) +}
\CommentTok{\#   geom\_histogram(stat = "identity", position = "dodge") +}
\CommentTok{\#   geom\_density(aes(color = tipos\_rações), size = 1.5) +  \# Adiciona a curva da densidade}
\CommentTok{\#   labs(title = "Distribuição de Turno por Tipos de Rações",}
\CommentTok{\#        x = "Turno",}
\CommentTok{\#        y = "Contagem",}
\CommentTok{\#        fill = "Tipos de Rações",}
\CommentTok{\#        color = "Tipos de Rações")  \# Adiciona a legenda para a curva da densidade}



\CommentTok{\# Assuming \textquotesingle{}produção\textquotesingle{} is the variable representing production in your dataset}

\CommentTok{\# Load necessary libraries}
\FunctionTok{library}\NormalTok{(ggplot2)}

\CommentTok{\# \# Create a histogram of production with density curve}
\CommentTok{\# ggplot(data = banco, aes(x = prod)) +}
\CommentTok{\#   geom\_histogram(binwidth = 1, fill = "lightblue", color = "black", alpha = 0.7) +  \# Adjust binwidth and transparency}
\CommentTok{\#   geom\_density(alpha = 0.8, fill = "\#FF6500") +  \# Add density curve with color and transparency}
\CommentTok{\#   labs(title = "Histograma da Produção", x = "Produção", y = "Densidade") +}
\CommentTok{\#   theme\_minimal() }


\CommentTok{\# ggplot(data = banco, aes(x = prod))+}
\CommentTok{\#   geom\_histogram() +  }
\CommentTok{\# geom\_density(position = "identity")}








\CommentTok{\# Calculate mean and standard deviation of your data}
\NormalTok{mean\_value }\OtherTok{\textless{}{-}} \FunctionTok{mean}\NormalTok{(banco}\SpecialCharTok{$}\NormalTok{prod)}
\NormalTok{sd\_value }\OtherTok{\textless{}{-}} \FunctionTok{sd}\NormalTok{(banco}\SpecialCharTok{$}\NormalTok{prod)}
\NormalTok{n }\OtherTok{\textless{}{-}} \FunctionTok{nrow}\NormalTok{(banco) }\CommentTok{\# Number of observations}
\NormalTok{lagura }\OtherTok{\textless{}{-}}\NormalTok{ binwidth }\OtherTok{\textless{}{-}}\DecValTok{1}  \CommentTok{\# Assuming you have defined binwidth earlier}

\NormalTok{autor }\OtherTok{\textless{}{-}} \StringTok{"joseferson Barreto"}
\NormalTok{ano }\OtherTok{\textless{}{-}}\StringTok{"2024"}
\CommentTok{\# Create a histogram of production with density curve and normal curve}
\FunctionTok{ggplot}\NormalTok{(}\AttributeTok{data =}\NormalTok{ banco, }\FunctionTok{aes}\NormalTok{(}\AttributeTok{x =}\NormalTok{ prod)) }\SpecialCharTok{+}
  \FunctionTok{geom\_histogram}\NormalTok{(}\AttributeTok{binwidth =}\NormalTok{ binwidth, }\AttributeTok{fill =} \StringTok{"lightblue"}\NormalTok{, }\AttributeTok{color =} \StringTok{"black"}\NormalTok{, }\AttributeTok{alpha =} \FloatTok{0.7}\NormalTok{) }\SpecialCharTok{+}
  \FunctionTok{geom\_density}\NormalTok{(}\AttributeTok{alpha =} \FloatTok{0.3}\NormalTok{, }\AttributeTok{fill =} \StringTok{"\#FFA500"}\NormalTok{) }\SpecialCharTok{+}
  \FunctionTok{stat\_function}\NormalTok{(}\AttributeTok{fun =} \ControlFlowTok{function}\NormalTok{(x) }\FunctionTok{dnorm}\NormalTok{(x, }\AttributeTok{mean =}\NormalTok{ mean\_value, }\AttributeTok{sd =}\NormalTok{ sd\_value) }\SpecialCharTok{*}\NormalTok{ n }\SpecialCharTok{*}\NormalTok{ lagura, }\AttributeTok{color =} \StringTok{"red"}\NormalTok{, }\AttributeTok{size =} \DecValTok{1}\NormalTok{) }\SpecialCharTok{+}
  \FunctionTok{labs}\NormalTok{(}\AttributeTok{title =} \StringTok{"Histograma da Produção com Curva Normal"}\NormalTok{, }\AttributeTok{x =} \StringTok{"Produção"}\NormalTok{, }\AttributeTok{y =} \StringTok{"Densidade"}\NormalTok{, }\AttributeTok{caption =} \FunctionTok{paste}\NormalTok{(}\StringTok{"Fonte: Elaborada pelo Autor,"}\NormalTok{, ano)) }\SpecialCharTok{+}
  \FunctionTok{theme}\NormalTok{(}\AttributeTok{panel.background =} \FunctionTok{element\_rect}\NormalTok{(}\AttributeTok{fill=}\StringTok{\textquotesingle{}\#a0a88a\textquotesingle{}}\NormalTok{, }\AttributeTok{colour=}\StringTok{\textquotesingle{}red\textquotesingle{}}\NormalTok{),}
        \AttributeTok{plot.caption =} \FunctionTok{element\_text}\NormalTok{(}\AttributeTok{hjust =} \FloatTok{0.5}\NormalTok{),}
        \AttributeTok{plot.title =} \FunctionTok{element\_text}\NormalTok{(}\AttributeTok{hjust =} \FloatTok{0.5}\NormalTok{))}
\end{Highlighting}
\end{Shaded}

\includegraphics{exploratory_files/figure-pdf/unnamed-chunk-7-1.pdf}

\begin{Shaded}
\begin{Highlighting}[]
\CommentTok{\#shapiro.test(banco$prod)}
\end{Highlighting}
\end{Shaded}

Como pode-se observar , a distribuição dos nossos dados apresenta um
achatamento maior que da distribuição normal, logo , podemos afirmar que
a variável \textbf{prod} não segue a distruição normal. outra forma de
observar isso , é pelo teste de Shapiro Wilk Shapiro e Wilk (1965)

\subsection{Teste De Shapiro Wilk}\label{teste-de-shapiro-wilk}

\subsubsection{Hipóteses do teste}\label{hipuxf3teses-do-teste}

\hfill\break

\begin{cases}
\text{H0:Os erros têm distribuição normal;   $\hspace{1,55cm}$            se $\alpha$ > 0,05} \\
\text{H1:Os erros não têm distribuição normal;  $\hspace{1cm}$        se  $\alpha$ < 0,05}
\end{cases}

\hfill\break

\begin{Shaded}
\begin{Highlighting}[]
\FunctionTok{shapiro.test}\NormalTok{(banco}\SpecialCharTok{$}\NormalTok{prod)}
\end{Highlighting}
\end{Shaded}

\begin{verbatim}

    Shapiro-Wilk normality test

data:  banco$prod
W = 0.9178, p-value < 2.2e-16
\end{verbatim}

Como o \(P\_{valor} < \alpha = 0,05\) temos que nossa variável não segue
a distribuição normal.

\section*{Referências}\label{referuxeancias}
\addcontentsline{toc}{section}{Referências}

\phantomsection\label{refs}
\begin{CSLReferences}{1}{0}
\bibitem[\citeproctext]{ref-shinythemes}
Chang, Winston. 2021. {«shinythemes: Themes for Shiny»}.
\url{https://CRAN.R-project.org/package=shinythemes}.

\bibitem[\citeproctext]{ref-shapiro1965}
Shapiro, S. S., e M. B. Wilk. 1965. {«An analysis of variance test for
normality (complete samples)»}. \emph{Biometrika} 52: 591--611.
\url{https://doi.org/10.2307/2333709}.

\bibitem[\citeproctext]{ref-ggplot2}
Wickham, Hadley. 2016. {«ggplot2: Elegant Graphics for Data Analysis»}.
\url{https://ggplot2.tidyverse.org}.

\bibitem[\citeproctext]{ref-readxl}
Wickham, Hadley, e Jennifer Bryan. 2023. {«readxl: Read Excel Files»}.
\url{https://CRAN.R-project.org/package=readxl}.

\bibitem[\citeproctext]{ref-dplyr}
Wickham, Hadley, Romain François, Lionel Henry, Kirill Müller, e Davis
Vaughan. 2023. {«dplyr: A Grammar of Data Manipulation»}.
\url{https://CRAN.R-project.org/package=dplyr}.

\bibitem[\citeproctext]{ref-knitr-2}
Xie, Yihui. 2023. {«knitr: A General-Purpose Package for Dynamic Report
Generation in R»}. \url{https://yihui.org/knitr/}.

\end{CSLReferences}



\end{document}
